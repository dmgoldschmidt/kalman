\documentclass[12pt,leqno]{article}
\usepackage{amsmath}
\usepackage{amsthm}
\usepackage{latexsym}
\usepackage{amsfonts}
\usepackage{amssymb}
\usepackage[all]{xy}
\usepackage{type1cm,times,mathptmx}
%\usepackage{type1cm}
%\usepackage[notcite, notref]{showkeys}
\newif\ifpdf
\ifx\pdfoutput\undefined
\pdffalse % we are not running PDFLaTeX
\else
\pdfoutput=1 % we are running PDFLaTeX
\pdftrue
\fi

\ifpdf
%put pdflatex-only commands here
\pdfcatalog{/OpenAction /Fith}
\usepackage[plainpages=false,pdfpagelabels, pdftex, colorlinks, bookmarks,bookmarksopen]{hyperref}
%\pdfcompresslevel=9

% prints ``DRAFT'' or whatever diagonally across each page
%\usepackage[pdftex]{graphics,color}                                          
%\usepackage{eso-pic}                                                         
%\AddToShipoutPicture{\resizebox{0.9\pdfpagewidth}{0.9\pdfpageheight}%       
%{\rotatebox{60}{\color[gray]{0.8}\hspace*{5mm}\textsc{Evaluation Copy}}}}

%another version of this
%\usepackage[pdftex]{graphics,color}                                          
%\usepackage{eso-pic} 
%\makeatletter
%  \AddToShipoutPicture{
%    \setlength{\@tempdimb}{.5\paperwidth}%
%    \setlength{\@tempdimc}{.5\paperheight}%
%    \setlength{\unitlength}{1pt}%
%    \put(\strip@pt\@tempdimb,\strip@pt\@tempdimc){%
%      \makebox(0,0){\rotatebox{45}{\textcolor[gray]{0.75}{\fontsize{5cm}{5cm}\selectfont{Evaluation Copy}}}}
%    }
%}
%\makeatother

\else
%put latex-only commands here
\usepackage[hypertex]{hyperref}
\fi
%\usepackage[pdftex, colorlinks, bookmarks,bookmarksopen, backref]{hyperref}
%\usepackage[pdftex,pagebackref]{hyperref}
%\usepackage[pagebackref]{hyperref}
%\usepackage{pst-all}
\newcommand{\re}{\ensuremath{{\frak Re}}}
\renewcommand{\P}{\ensuremath{{\mathbb P}}}
\renewcommand{\O}{\ensuremath{{\cal O}}}
\renewcommand{\S}{\ensuremath{{\cal S}}}
\renewcommand{\L}{\ensuremath{{\cal L}}}
\newcommand{\J}{\ensuremath{{\mathbf{J}}}}
\newcommand{\bS}{\ensuremath{{\bar{S}}}}
\newcommand{\I}{\ensuremath{{\cal I}}}
\newcommand{\Y}{\ensuremath{{\cal Y}}}
\newcommand{\V}{\ensuremath{{\cal V}}}
\newcommand{\A}{\ensuremath{{\mathbb A}}}
\newcommand{\Z}{\ensuremath{{\mathbb Z}}}
\newcommand{\R}{\ensuremath{{\mathbb R}}}
\newcommand{\Q}{\ensuremath{{\mathbb Q}}}
\newcommand{\C}{\ensuremath{{\mathbb C}}}
\newcommand{\N}{\ensuremath{{\mathbb N}}}
\newcommand{\e}{\ensuremath{{\epsilon}}}
\newcommand{\f}{\ensuremath{{\mathfrak f}}}
\newcommand{\F}{\ensuremath{{\mathfrak F}}}
\newcommand{\FF}{\ensuremath{{\cal F}}}
\newcommand{\G}{\ensuremath{{\cal G}}}
\newcommand{\LD}{\ensuremath{{\cal L}}}
\newcommand{\E}{\ensuremath{{\cal E}}}
\newcommand{\M}{\ensuremath{{\cal M}}}
\newcommand{\NN}{\ensuremath{{\cal N}}}
%\newcommand{\qed}{\ensuremath{~\Box}\vspace{.1in}}
\newcommand{\ab}{\ensuremath{\text{\rm ab}}}
\newcommand{\hgt}{\ensuremath{\text{\rm ht}}}
\newcommand{\abel}{\ensuremath{\text{\rm abel}}}
\newcommand{\rank}{\ensuremath{\text{\rm rank}}}
\newcommand{\tr}{\ensuremath{\text{\rm tr}}}
\newcommand{\lcm}{\ensuremath{\text{\rm lcm}}}
\newcommand{\sep}{\;{\rm sep}}
\newcommand{\diag}{\ensuremath{\text{\rm diag}}}
\newcommand{\ch}{\ensuremath{\text{\rm char}}}
\newcommand{\HOM}{\ensuremath{\mathbf{Hom}}}
\newcommand{\Hom}{\ensuremath{\text{\rm Hom}}}
\newcommand{\Ext}{\ensuremath{\text{\rm Ext}}}
\newcommand{\End}{\ensuremath{\text{\rm End}}}
\newcommand{\trdeg}{\ensuremath{\text{\rm trdeg}}}
\newcommand{\Ann}{\ensuremath{\text{\rm Ann}}}
\newcommand{\lub}{\ensuremath{\text{\rm lub}}}
\newcommand{\Res}{\ensuremath{\text{\rm Res}}}
\newcommand{\im}{\ensuremath{\text{\rm im}}}
\newcommand{\init}{\ensuremath{\text{\rm in}}}
\newcommand{\gr}{\ensuremath{\text{\rm gr}}}
\newcommand{\coker}{\ensuremath{\text{\rm coker}}}
\newcommand{\codim}{\ensuremath{\text{\rm codim}}}
\newcommand{\D}{\ensuremath{{\cal D}}}
\newcommand{\result}{\ensuremath{{\cal R}}}
\newcommand{\Div}{\ensuremath{\text{\rm Div}}}
\newcommand{\Gal}{\ensuremath{\text{\rm Gal}}}
\newcommand{\Aut}{\ensuremath{\text{\rm Aut}}}
\newcommand{\supp}{\ensuremath{\text{\rm supp}}}
\newcommand{\ilim}{\ensuremath{\overset{\leftarrow}{\lim}}}
\newcommand{\dlim}{\ensuremath{\underset{\rightarrow}{\lim}}}
\newcommand{\bV}{\ensuremath{\mathbf{V}}}
\newcommand{\bI}{\ensuremath{\mathbf{I}}}
\newcommand{\Sp}{\ensuremath{\text{\rm Specm}}}
\newcommand{\Sh}{\ensuremath{\text{\rm Sh}}}
\newcommand{\normal}{\ensuremath{\triangleleft\;}}
\newcommand{\subnormal}{\ensuremath{\triangleleft\triangleleft\;}}
%\theoremstyle{break}
\makeatletter\renewcommand\theequation{\@arabic\c@equation}\makeatother
\numberwithin{equation}{section}
\newtheorem{Thm}[equation]{Theorem}
\newtheorem{Lem}[equation]{Lemma}
\newtheorem{Cor}[equation]{Corollary}
\newtheorem{Hyp}[equation]{Hypothesis}
\theoremstyle{definition}
\newtheorem{Ex}{Exercise}
%\numberwithin{Ex}{chapter}
\let\theHEx\theEx
\newtheorem*{Def}{Definition}

\title{A Continuous Hidden Markov Model}
\author{David M. Goldschmidt}
%\oddsidemargin 10 pt \evensidemargin 10 pt \marginparwidth 0.75 in \textwidth
%6.0 true in \topmargin -40 pt \textheight 8.8 true in 
%\usepackage{fancyhdr}
%\pagestyle{fancy}
%\lfoot{}
%\rfoot{\thepage}
\begin{document}
%\renewcommand{\footrulewidth}{0.4pt}
\newcommand{\p}{\ensuremath{u}}
\newcommand{\VV}{V}
\maketitle

\section{Introduction}
We define an HMM with state space $(-\infty,\infty)$ as follows.  Given state $s$ and observation $x$ at time
$t$, , the (gaussian) output density is
$$
X(x\mid s) := \frac{1}{\sigma_x\sqrt{2\pi}}\exp-\frac{1}{2}\left(\frac{(x-s)^2}{\sigma_x^2}\right)dx.
$$
The state process is a discrete time continuous state Markov process with transition function
from state $s_0$ at time $t$ to state $s_1$ at time $t+1$ given by
$$
Pr(s_1\mid s_0) :=  \frac{1}{\sigma_0\sqrt{2\pi}}\exp-\frac{1}{2}\left(\frac{(s_1-s_0)^2}{\sigma_0^2}\right).
$$
We see that $s$ is the mean of the output density, while $\sigma_x$ and $\sigma_0$ are time-independent
parameters of the model. Thus, the output distribution is i.i.d. gaussian.

As usual, we are given observations $\{x_t\mid 1\le t\le T\}$ and model parameters $\theta$.
For each observation time $t$ and state $s$, we define
\begin{align*}
  \alpha_t(s) :&= Pr(x_1,x_2,\dots,x_t~ \& \text{state $s$ at time $t$}\mid \theta ),\\
  \beta_t(s) :&= Pr(x_{t+1},\dots,x_T \mid ~\text{state $s$ at time $t$},\theta)\\
\end{align*}

It is clear from the definitions that the following recursions are satisfied:
\begin{align}
\alpha_t(s) &= X(x_t \mid s)\int_{-\infty}^{\infty}\alpha_{t-1}(u)Pr(s \mid u)du,\quad\text{and}\label{alpha:0}\\
\beta_t(s) &= \int_{-\infty}^{\infty}Pr(u \mid s)X(x_{t+1} \mid u)\beta_{t+1}(u)du.\label{beta:0}
\end{align}

We initialize these recursions with
\begin{align*}
\alpha_0(s) :&= \frac{1}{\sigma_{a,0}\sqrt{2\pi}}\exp-\frac{1}{2}\left(\frac{(s-\mu_{a,0})^2}{\sigma_{a,0}^2}\right),\\
\beta_{T+1}(s) :&= 1 \quad\text{for all $s$},
\end{align*}
where $\sigma_{a,0}$ and $\mu_{a,o}$ are model parameters.

Because everything in sight is gaussian, the above integrals can be evaluated in closed form, by repeated application of

\begin{Lem}\label{comp_sq}
  Let
  $$
  Q(x) := \frac{(x-\mu_1)^2}{\sigma_1^2} + \frac{(x-\mu_2)^2}{\sigma_2^2}.
  $$
  then
\begin{align}
    Q(x) &= \frac{(x-\mu)^2}{\sigma^2} + \frac{(\mu_1-\mu_2)^2}{\sigma_1^2+\sigma_2^2},\quad\text{where}\notag\\
    \mu :&= \frac{\sigma_2^2\mu_1 + \sigma_1^2\mu_2}{\sigma_1^2+\sigma_2^2},\quad\text{and}\label{cs:mu}\\
    \sigma^2 :&= \frac{\sigma_1^2\sigma_2^2}{\sigma_1^2+\sigma_2^2}.\label{cs:sigma}
\end{align}
In particular, the combined mean $\mu$ is a convex linear combination of $\mu_1$ and $\mu_2$ with weights proportional to $\sigma_2^2$ and $\sigma_1^2$ respectively, while the combined variance $\sigma^2$ satifies $\sigma^{-2} = \sigma_1^{-2} + \sigma_2^{-2}$.
\end{Lem}
  \begin{proof}
    This is basically an exercise in completing the square.  We have

\begin{align*}
  Q(x) &= \sigma_1^{-2}(x-\mu_1)^2 + \sigma_2^{-2}(x-\mu_2)^2\\
  &= (\sigma_1^{-2}+\sigma_2^{-2})x^2 - 2(\sigma_1^{-2}\mu_1+\sigma_2^{-2}\mu_2)x + \sigma_1^{-2}\mu_1^2+\sigma_2^{-2}\mu_2^2\\
  &= a^2x^2 -2bx +c = (ax-a^{-1}b)^2 + c - a^{-2}b^2 = a^2(x-a^{-2}b)^2 + c - a^{-2}b^2 \\
\end{align*}
where
$$
a^2 = \sigma_1^{-2} + \sigma_2^{-2},\quad b = \sigma_1^{-2}\mu_1 + \sigma_2^{-2}\mu_2,\quad\text{and}\quad
c = \sigma_1^{-2}\mu_1^2+\sigma_2^{-2}\mu_2^2.
$$
Then
\begin{align*}
  a^2(x-a^{-2}b^2) &= (\sigma_1^{-2}+\sigma_2^{-2})\left(x - \frac{\sigma_1^{-2}\mu_1+\sigma_2^{-2}\mu_2}{\sigma_1^{-2}+\sigma_2^{-2}}\right)^2 \\
  &= \frac{\sigma_1^2 +\sigma_2^2}{\sigma_1^2\sigma_2^2}\left(x - \frac{\sigma_2^2\mu_1+\sigma_1^2\mu_2}{\sigma_1^2+\sigma_2^2}\right)^2\\
  &= \frac{(x-\mu)^2}{\sigma^2},\quad\text{as required, and}\\
  c-a^{-2}{b^2} &= \sigma_1^{-2}\mu_1^2+\sigma_2^{-2}\mu_2^2 - \frac{(\sigma_1^{-2}\mu_1 + \sigma_2^{-2}\mu_2)^2}
  {\sigma_1^{-2}+\sigma_2^{-2}} \\
  &=\frac{\sigma_1^{-4}\mu_1^2+\sigma_1^{-2}\sigma_2^{-2}(\mu_1^2+\mu_2^2)+\sigma_2^{-4}\mu_2^2 -
    \sigma_1^{-4}\mu_1^2 - 2\sigma_1^{-2}\sigma_2^{-2}\mu_1\mu_2 -\sigma_2^{-4}\mu_2^2}{\sigma_1^{-2}+\sigma_2^{-2}}\\
  &= \frac{\sigma_1^{-2}\sigma_2^{-2}(\mu_1^2 - 2\mu_1\mu_2+\mu_2^2)}{\sigma_1^{-2}+\sigma_2^{-2}} \\
  &= \frac{(\mu_1-\mu_2)^2}{\sigma_1^2+\sigma_2^2}.
  \end{align*}
  \end{proof}

  \section{The Alpha Pass}

  Now we can inductively evaluate \eqref{alpha:0}. To do so, we split the computation
  into two steps.  In the first step, which we call the {\em time update}, we multiply
  $\alpha_{t-1}(u)$ by the state transition function $Pr(s\mid u)$ and integrate with respect
  to $u$. We will denote the result of the time update by $\hat{\alpha}_t(s)$.  It is the joint probability
  (density) of observations $x_1,\dots,x_{t-1}$ and state $s$ at time $t$.
  Then in the second step, which we call the {\em measurement update}, we multiply
  $\hat{\alpha}_t(s)$ by the probability (density) of observing $x_t$ at time $t$ to get $\alpha_t(s)$.

  First, we define
$$
P_{m,n} := Pr(x_m,x_{m+1},\dots,x_n \mid \theta),\quad\text{for $1\le m \le n\le T$}.
$$
\begin{Thm}\label{alpha:1}
  For each state $s$ and time $t$,
$$
  \alpha_t(s) = \frac{P_{1,t}}{\sigma_{a,t}\sqrt{2\pi}}\exp-\frac{1}{2}\left(\frac{(s-\mu_{a,t})^2}{\sigma_{a.t}^2}\right),
$$
  where

  \begin{align*}
  \sigma_{a,t}^2 &= \frac{\hat{\sigma}_{a,t}^2\sigma_x^2}{\hat{\sigma}_{a,t}^2+\sigma_x^2}\\
  \mu_{a,t} &= \frac{\hat{\sigma}_{a,t}^2\mu_x + \sigma_x^2\hat{\mu}_{a,t-1}}{\hat{\sigma}_{a,t}^2+\sigma_x^2}\quad\text{and}\\
  P_{1,t} &=
\end{align*}
\end{Thm}

\begin{proof}

We proceed by induction on $t$, noting that the case $t = 0$ holds by definition.
For the time update, we let
\begin{align*}
  \hat{\mu}_{a,t} &:= \frac{\sigma_0^2\mu_{a,t-1}}{\sigma_{a,t-1}^2 + \sigma_0^2},\\
  \hat{\sigma}_{a,t}^2 &:=\sigma_{a,t-1}^2\sigma_0^2, \quad\text{and}\\
  \rho^2 &:= \frac{\sigma_{a,t-1}^2\sigma_0^2}{\hat{\sigma}_{a,t}^2}.
\end{align*}
Then using \eqref{comp_sq} we have 
\begin{equation}\label{alpha:2}
  \begin{split}
  \hat{\alpha}_t(s) &= \frac{P_{1,t-1}}{2\pi\sigma_{a,t-1}\sigma_0}
  \int_{-\infty}^{\infty}\exp-\frac{1}{2}\left(\frac{(u-\mu_{a,t-1})^2}{\sigma_{a,t-1}^2} + \frac{(u-s)^2}{\sigma_0^2}\right)du \\
  &= \frac{P_{1,t-1}}{2\pi\sigma_{a,t-1}\sigma_0}
  \int_{-\infty}^{\infty}\exp-\frac{1}{2}\left(\frac{(u-\hat{\mu}_{a,t})^2}{\rho^2} +
  \frac{(s - \mu_{a,t-1})^2}{\hat{\sigma}_{a,t}^2}\right)du \\
      &= \frac{P_{1,t-1}}{2\pi\sigma_{a,t-1}\sigma_0x} \exp-\frac{1}{2}\left(\frac{(s-\mu_{a,t-1})^2}{\hat{\sigma}_{a,t}^2}\right)
  \int_{-\infty}^{\infty}\exp-\frac{1}{2}\left(\frac{(u-\hat{\mu}_{a,t-1})^2}{\rho^2} \right)du \\
  &= \frac{P_{1,t-1}\rho}{\sqrt{2\pi}\sigma_{a,t-1}\sigma_0}\exp-\frac{1}{2}\left(\frac{(s-\mu_{a,t-1})^2}{\hat{\sigma}_{a,t}^2}\right)\\
  &= \frac{P_{1,t-1}}{\sqrt{2\pi}\hat{\sigma}_{a,t}}\exp-\frac{1}{2}\left(\frac{(s-\mu_{a,t-1})^2}{\hat{\sigma}_{a,t}^2}\right).
  \end{split}
  \end{equation}

We now apply the measurement update, and using \eqref{comp_sq} once more, we get
\begin{align*}
  \alpha_t(s) &= \frac{P_{1,t-1}}{2\pi\hat{\sigma}_{a,t}\sigma_x}
  \exp-\frac{1}{2}\left(\frac{(s-\mu_{a,t-1})^2}{\hat{\sigma}_{a,t}^2} + \frac{(s-x_t)^2}{\sigma_x^2}\right)\\
  &= \frac{P_{1,t-1}}{2\pi\hat{\sigma}_{a,t}\sigma_x}
  \exp-\frac{1}{2}\left(\frac{(s-\mu_{a,t})^2}{\sigma_{a,t}^2 } +
  \frac{(\mu_{a,t-1}-x_t)^2}{\hat{\sigma}_{a,t}^2 +\sigma_x^2}\right)\\
  &= \frac{P_{1,t-1}\sigma_{a,t}}{\sqrt{2\pi}\hat{\sigma}_{a,t}\sigma_x}\exp-\frac{1}{2}\left(\frac{(\mu_{a,t-1}-x_t)^2}{\hat{\sigma}_{a,t}^2 +\sigma_x^2}\right)\frac{1}{\sqrt{2\pi}\sigma_{a,t}}\exp-\frac{1}{2}\left(\frac{(s-\mu_{a,t})^2}{\sigma_{a,t}^2 }
  \right),
\end{align*}
where, using \eqref{cs:mu}and \eqref{cs:sigma}, we get
\begin{align*}
  \mu_{a,t} &= \frac{\hat{\sigma}_{a,t}^2\mu_x + \sigma_x^2\mu_{a,t-1}}{\hat{\sigma}_{a,t}^2+\sigma_x^2}\\
  &= \frac{(\sigma_{a,t-1}^2+\sigma_0^2)\mu_x + \sigma_x^2\mu_{a,t-1}}{\sigma_{a,t-1}^2+\sigma_0^2+\sigma_x^2}, \\ 
  \sigma_{a,t}^2 &= \frac{\hat{\sigma}_{a,t}^2\sigma_x^2}{\hat{\sigma}_{a,t}^2 + \sigma_x^2} \\
  &= \frac{(\sigma_{a,t-1}^2+\sigma_0^2)\sigma_x^2}{\sigma_{a,t-1}^2+\sigma_0^2+\sigma_x^2}, \quad\text{and}\\
  P_{1,t}&= \frac{P_{1,t-1}\sigma_{a,t}}{\sqrt{2\pi}\hat{\sigma}_{a,t}\sigma_x}\exp-\frac{1}{2}\left(\frac{(\mu_{a,t-1}-x_t)^2}{\hat{\sigma}_{a,t}^2 +\sigma_x^2}\right)\\
%  &=  \frac{P_{1,t-1}\sigma_0\hat{\sigma}_{a,t}\sigma_x}{\sqrt{\hat{\sigma}_{a,t}^2+\sigma_x^2}\hat{\sigma}_{a,t}\sigma_x}\exp-\frac{1}{2}\left(\frac{(\mu_{a,t-1}-x_t)^2}{\hat{\sigma}_{a,t}^2 +\sigma_x^2}\right) \\
  &= \frac{P_{1,t-1}\sigma_0}{\sqrt{2\pi(\sigma_{a,t-1}^2+\sigma_0^2+\sigma_x^2)}}\exp-\frac{1}{2}\left(\frac{(\mu_{a,t-1}-x_t)^2}
    {\sigma_{a,t-1}^2+\sigma_0^2 +\sigma_x^2}\right).
\end{align*}
\end{proof}  


\section{Introduction}
We define an HMM with state space $\R^n$ as follows.  Given state $s\in\R^n$ and an observation $x\in\R^n$ at time
$t$, the (gaussian) output density is
$$
X(x\mid s) := ({2\pi}|\Sigma_x|)^{-\frac{n}{2}}\exp\left\{-\frac{1}{2}
(x-s)^T\Sigma_x^{-1}(x-\mu)\right\}.
$$
The state process is a discrete time continuous state Markov process with transition probability density
from state $s_0$ at time $t$ to state $s_1$ at time $t+1$ given by
$$
Pd(s_1\mid s_0) :=  ({2\pi}|\Sigma_0|)^{-\frac{n}{2}}\exp\left\{-\frac{1}{2}(s_1-s_0)^T\Sigma_0^{-1}(s_1-s_0)\right\}.
$$
We see that $s$ is the mean of the output density, while $\Sigma_x$ and $\Sigma_0$ are time-independent
parameters of the model. Thus, the output distribution is i.i.d. gaussian.

As usual, we are given observations $\{x_t\mid 1\le t\le T\}$ and model parameters $\theta$.  For each observation time $t$ and state $s$, we define
\begin{align*}
  \alpha_t(s) :&= Pd(x_1,x_2,\dots,x_t~ \& \text{state $s$ at time $t$}\mid \theta ),\\
  \beta_t(s) :&= Pr(x_{t+1},\dots,x_T \mid ~\text{state $s$ at time $t$},\theta)\\
\end{align*}

It is clear from the definitions that the following recursions are satisfied:
\begin{align}
\alpha_t(s) &= X(x_t \mid s)\int_{\R^n}\alpha_{t-1}(u)Pd(s \mid u)du,\quad\text{and}\label{alpha:0}\\
\beta_t(s) &= \int_{\R^{n}}Pd(u \mid s)X(x_{t+1} \mid u)\beta_{t+1}(u)du.\label{beta:0}
\end{align}

We initialize these recursions with
\begin{align*}
\alpha_0(s) :&= (2\pi|\Sigma_{a,0}|)^{-\frac{n}{2}}\exp\left\{-\frac{1}{2}(s-\mu_{a,0})^T\Sigma_{a,0}^{-1}(s-\mu_{a,0})\right\},\\
\beta_{T+1}(s) :&= 1 \quad\text{for all $s$},
\end{align*}
where $\Sigma_{a,0}$ and $\mu_{a,o}$ are model parameters.

Because everything in sight is gaussian, the above integrals can be evaluated in closed form, by repeated application of
\begin{Lem}\label{comp_sq:1}
  Let
  $$
  Q_i(x) := (x-\mu_i)^T\Sigma_i^{-1}(x-\mu_i)\quad(i = 1,2),
  $$
  where $x$ and $\mu_i$ are $n$-dimensional vectors and $\Sigma_i$ is a symmetric positive definite $n\times{n}$
  matrix ($i = 1,2)$. Then
  $$
  Q_1(x) + Q_2(x) = (x-\mu)^T\Sigma^{-1}(x-\mu) + (\mu_2-\mu_1)^T(\Sigma_1+\Sigma_2)^{-1}(\mu_2-\mu_1),
  $$
  where
  \begin{align*}
    \mu :&= \Sigma_2(\Sigma_1+\Sigma_2)^{-1}\mu_1+\Sigma_1(\Sigma_1+\Sigma_2)^{-1}\mu_2,\quad\text{and}\\
    \Sigma^{-1} :&= \Sigma_1^{-1} + \Sigma_2^{-1}.
  \end{align*}
\end{Lem}
\begin{proof}

It's straightforward to see that
$$
Q_1(x) + Q_2(x) = (x-\mu)^T(\Sigma^{-1})(x-\mu) + C,
$$
where
\begin{align}
 \Sigma^{-1} &= \Sigma_1^{-1} + \Sigma_2^{-1},\label{Sigma}\\
  \Sigma^{-1}\mu &=\Sigma_1^{-1}\mu_1+\Sigma_2^{-1}\mu_2, \quad\text{and}\label{mu}\\
 C &= \mu_1^T\Sigma_1^{-1}\mu_1 +\mu_2^T\Sigma_2^{-1}\mu_2 - \mu^T\Sigma^{-1}\mu.\label{C}
\end{align}
The problem is to put $C$ into the form stated in the lemma.
 To begin with, we have

\begin{align}
  \Sigma_1\Sigma^{-1}\Sigma_2 &= \Sigma_1(\Sigma_1^{-1}+\Sigma_2^{-1})\Sigma_2 = \Sigma_1+\Sigma_2
  = \Sigma_2\Sigma^{-1}\Sigma_1,\quad\text{whence}\notag\\
  \Sigma_1^{-1}\Sigma\Sigma_2^{-1} &= (\Sigma_1+\Sigma_2)^{-1} = \Sigma_2^{-1}\Sigma\Sigma_1^{-1},\quad\text{and thus}\notag\\
  \Sigma &= \Sigma_1(\Sigma_1 + \Sigma_2)^{-1}\Sigma_2 = \Sigma_2(\Sigma_1+\Sigma_2)^{-1}\Sigma_1,\label{Sigma:1}
\end{align}
Then \eqref{mu} becomes
\begin{equation}\label{mu:1}
  \begin{split}
    \mu &=  \Sigma(\Sigma_1^{-1}\mu_1+\Sigma_2^{-1}\mu_2)\\
    &=\Sigma_2(\Sigma_1+\Sigma_2)^{-1}\mu_1+\Sigma_1(\Sigma_1+\Sigma_2)^{-1}\mu_2,
  \end{split}
\end{equation}
 But we can re-write \eqref{mu:1} as  
\begin{equation}\label{mu:2}
  \begin{split}
  \mu &= \Sigma(\Sigma_1^{-1}\mu_1 + \Sigma_2^{-1}\mu_1 +\Sigma_2^{-1}(\mu_2-\mu_1)) \\
  &= \Sigma(\Sigma^{-1}\mu_1 + \Sigma_2^{-1}(\mu_2-\mu_1) \\
  &= \mu_1 + \Sigma_1(\Sigma_1+\Sigma_2)^{-1}(\mu_2-\mu_1),
  \end{split}
  \end{equation}
and by symmetry, we also have 
\begin{equation}\label{mu:3}
  \mu = \mu_2 + \Sigma_2(\Sigma_1+\Sigma_2)^{-1}(\mu_1-\mu_2).
\end{equation}
Then we premultiply \eqref{mu} by $\mu^T$, substituting \eqref{mu:2} into the first term and
  \eqref{mu:3} into the second as well as using \eqref{Sigma:1}, to get 
\begin{align*}
    \mu^T\Sigma^{-1}\mu &= \mu^T\Sigma_1^{-1}\mu_1 + \mu^T\Sigma_2^{-1}\mu_2 \\
    &= \mu_1^T\Sigma_1^{-1}\mu_1 + (\mu_2-\mu_1)^T(\Sigma_1+\Sigma_2)^{-1}\mu_1
    +\mu_2^T\Sigma_1\mu_2 + (\mu_1-\mu_2)^T(\Sigma_1+\Sigma_2)^{-1}\mu_2\\
    &= \mu_1^T\Sigma_1^{-1}\mu_1 + \mu_2^T\Sigma_2^{-1}\mu_2 - (\mu_2-\mu_1)^T(\Sigma_1+\Sigma_2)^{-1}(\mu_2-\mu_1).
\end{align*}

Finally, substituting this result into \eqref{C} yields
\begin{equation}\label{C:1}
  C = (\mu_2-\mu_1)^T(\Sigma_1+\Sigma_2)^{-1}(\mu_2-\mu_1)
\end{equation}
as required.
\end{proof}
\end{document}



